\documentclass[a4paper,12pt]{article}
\usepackage[utf8]{inputenc}
\usepackage[russian]{babel}
\usepackage{amsmath}
\usepackage{graphicx}
\usepackage{hyperref}

\linespread{1.3}

\title{Пример \LaTeX\ документа}
\author{Халаманов Алексей}
\date{\today}

\begin{document}

\maketitle

\renewcommand{\abstractname}{}
\vspace{-4em}

\begin{abstract}
    \textbf{Аннотация} --- (от лат. \textit{annotatio} «замечание»; или \textbf{резюме}) (от фр. \textit{résumé} «сокращённый»; или англ. Summary «сводка») — краткое содержание книги, рукописи, монографии, статьи, патента, фильма, грампластинки или другого издания, а также его краткая характеристика. 
\end{abstract}

\section*{Введение}
    \LaTeX\ является мощным инструментом для подготовки научных и технических документов. В этом разделе представлены основные функции \LaTeX{}.

\section{Абзацы и списки}

\subsection{Абзацы}
    Тексты в \LaTeX\ автоматически разбиваются на абзацы, когда остаётся пустая строка между текстом.

    Это первый абзац. Первый, первый,  первый,  первый,  первый,  первый,  первый,  первый,  первый,  первый,  первый, первый,  первый, первый.

    Это второй абзац. Второй, второй, второй, второй, второй, второй, второй, второй, второй, второй, второй, второй, второй, второй, второй.
\newpage

\subsection{Нумерованные и маркированные списки}

    Нумерованный список:
    \begin{enumerate}
        \item Первый пункт
        \item Второй пункт
        \item Третий пункт
    \end{enumerate}

    Маркированный список:
    \begin{itemize}
        \item Первый элемент
        \item Второй элемент
        \item Третий элемент
    \end{itemize}

\section{Математические выражения}
    Например, квадратное уравнение: $ax^2 + bx+ c = 0$ или вычисление дискриминанта:
    \[D=b^2-4ac\]

    Можно делать окружение, в котором уравнение автоматически нумеруется. Используя команду "equation":
    \begin{equation}
        ax^3 + bx^2+cx+d=0
    \end{equation}
    \label{eq:triple}
    \begin{equation}
        \text{где } a \neq 0
    \end{equation}
    С помощью "allign"\ можно делать многострочное уравнение с выравниеванием по определённому символу, например по знаку равенства:
    \begin{align}
        \int_a^b f(x) \, dx &= F(b) - F(a) \\
        E &= mc^2 \\
        2 + 2 &= 4 \\
        log_{2}(8) &= 3 \\
        100 &= 10^2
    \end{align}

\section{Таблицы}
    Для табулирования текстовой информации, используем  окружение tabbing:

    \begin{tabbing}
        ЖЖЖ \= ЖЖЖЖЖЖЖЖЖЖЖЖЖ \= ЖЖЖЖЖЖЖЖ\= ЖЖЖЖЖ \kill
        № \>  ФИО \>  Возраст \>  Рост (см) \\
        1 \> Иванов И. И. \> 13 \> 165 \\
        2 \> Петров П. П. \> 45 \>  402\\
        3 \> Сидоров С. С. \> 72 \> 87
    \end{tabbing}

    Или вот так:

    \begin{tabbing}
        \hspace{3em}\= \hspace{12em}\= \hspace{6em}\= \hspace{6em}\kill
        № \>  ФИО \>  Возраст \>  Рост (см) \\
        1 \> Иванов И. И. \> 13 \> 165 \\
        2 \> Петров П. П. \> 45 \>  402\\
        3 \> Сидоров С. С. \> 72 \> 87
    \end{tabbing}

    Можно использовать tabular, тогда будут границы: \\

    \begin{tabular}{l||l||cc}
        \hline
        № & ФИО & Возраст & Ротс (см)\\
        \hline\hline
        1 & Иванов И. И. & 13 & 165 \\
        2 & Петров П. П. & 45 & 402 \\
        3 & Сидоров С. С. & 72 & 87
    \end{tabular}
\newpage
    Обычная таблица с помощью table и tabular:

    \begin{table}[h]
        \centering
        \begin{tabular}{|c|c|c|}
            \hline
            Колонка 1 & Колонка 2 & Колонка 3 \\
            \hline\hline
            1 & 2 & 3 \\
            \hline
            a & b & c \\
            \hline
        \end{tabular}
        \caption{Пример таблицы с подписью}
        \label{tab:example_table}
    \end{table}

\section{Рисунки}
Вставим своё изображение:

\begin{figure}[h]
    \centering
    \includegraphics[width=0.8\textwidth]{pic_1.jpg}
    \caption{Тяжёлое положение электрика}
    \label{fig:pic_1}
\end{figure}

\section*{Ссылки}
    Например, ссылка на таблицу \ref{tab:example_table}, на рисунок \ref{fig:pic_1}, и на уравнение \ref{eq:triple}. Например, вот ссылка на сайт \href{https://www.latex-project.org/}{LaTeX Project}.

\end{document}